%++++++++++++++++++++++++++++++++++++++++
% Don't modify this section unless you know what you're doing!
\documentclass[letterpaper,12pt]{article}
\usepackage{tabularx} % extra features for tabular environment
\usepackage{amsmath}  % improve math presentation
\usepackage{graphicx} % takes care of graphic including machinery
\usepackage[margin=1in,letterpaper]{geometry} % decreases margins
\usepackage{cite} % takes care of citations
\usepackage[final]{hyperref} % adds hyper links inside the generated pdf file
\usepackage[numbib,nottoc]{tocbibind}
\usepackage{float}
\hypersetup{
	colorlinks=true,       % false: boxed links; true: colored links
	linkcolor=blue,        % color of internal links
	citecolor=blue,        % color of links to bibliography
	filecolor=magenta,     % color of file links
	urlcolor=blue         
}
%++++++++++++++++++++++++++++++++++++++++


\begin{document}

\title{PDA Kaggle Project}
\author{Andreas Vieider (13177)}
\date{\today}
\maketitle

\begin{abstract}
In this project I tested different implementations of the SVD algorithm for recommending movies. 
The libraries used are Scipy and Tensorflow.
\end{abstract}


\section{Introduction}
The idea for using Scipy was taken from the tutorial ~\cite{Scipy} of Nick Becker, whereas the description for using 
Tensorflow's SVD for recommending comes from Sujit Pal ~\cite{Tensorflow}.



\section{Theory}
Both implementations are based on model-based collaborative filtering. SVD (Singular Value Decomposition) takes the matrix A and decomposes it into three matrices, namely the user features U, item features V\textsuperscript{t} and the diagonal matrix of singular values \( \Sigma \). A can be re-constructed by multiplying the three matrices.



\section{Procedures}
To find the optimal number of features k, the RMSE and the variance are computed with Tensorflow.


\section{Implementation}
The inputs (shown in Figure~\ref{fig:input}) are, besides the mandatory ones like Pandas and Numpy, svds and mean squared error from Scipy and tensorflow from Tensorflow.
Datetime is used for labeling the output file with the current time.

\begin{figure}[H] 
  \centering
      \includegraphics[width=\columnwidth]{figures/input.PNG}
        \caption{
                \label{fig:input}  
                Input
        }
\end{figure}

The first step is to read in the data, which is done with Pandas. The ratings are then transformed into a matrix user*movies. This is shown in Figure~\ref{fig:data}. For testing purposes ratings also have been de-biased, regarding global-, user- and item-biases (Figure~\ref{fig:biases})

\begin{figure}[H] 
  \centering
      \includegraphics[width=\columnwidth]{figures/data.PNG}
        \caption{
                \label{fig:data}  
                Data
        }
\end{figure}

\begin{figure}[H] 
  \centering
      \includegraphics[width=\columnwidth]{figures/biases.PNG}
        \caption{
                \label{fig:biases}  
                De-biased Ratings
        }
\end{figure}

The use of the Scipy implementation is straight forward and shown in Figure~\ref{fig:svd1}. On the other hand, for Tensorflow you need to use Graph and Session (as in Figure~\ref{fig:svd2}) to exploit the functionality.

\begin{figure}[H] 
  \centering
      \includegraphics[width=\columnwidth]{figures/svd1.PNG}
        \caption{
                \label{fig:svd1}  
                Scipy SVD
        }
\end{figure}

\begin{figure}[H] 
  \centering
      \includegraphics[width=\columnwidth]{figures/svd2.PNG}
        \caption{
                \label{fig:svd2}  
                Tensorflow SVD
        }
\end{figure}

Finally, to recommend the top ten movies the function shown in Figure~\ref{fig:recommend} is called for each user.

\begin{figure}[H] 
  \centering
      \includegraphics[width=\columnwidth]{figures/recommend.PNG}
        \caption{
                \label{fig:recommend}  
                Recommend movies for each user
        }
\end{figure}


\section{Analysis}
Although, the two algorithms score the highest in terms of MAP@10, unfortunately the matrix is too sparse to predict the already given ratings.
Therefore, I used the approach of Sujit Pal ~\cite{Tensorflow} and computed the RMSE and R\textsuperscript{2} with Tensorflow (Figure~\ref{fig:evaluate}).

\begin{figure}[H] 
  \centering
      \includegraphics[width=\columnwidth]{figures/evaluate.PNG}
        \caption{
                \label{fig:evaluate}  
                Evaluation for k and biases
        }
\end{figure}

\begin{figure}[H] 
  \centering
      \includegraphics{figures/outcome.PNG}
        \caption{
                \label{fig:outcome}  
                Evaluation result
        }
\end{figure}

\section{Conclusions}
In conclusion, the best solution was submitted using k=15 and without de-biasing with a score of about 0.079.

\bibliography{references}{}
\bibliographystyle{plain}


\end{document}
